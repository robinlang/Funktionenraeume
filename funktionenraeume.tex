\documentclass[
paper=a4,
bibtotocnumbered,
liststotocnumbered,
tablecaptionabove,
pointlessnumbers,
twoside,
openright,
10pt
]
{report}  
\usepackage{mathtools}
\mathtoolsset{showonlyrefs}

\usepackage{etex}
\usepackage[all]{xy}
\usepackage[english, ngerman]{babel}
\usepackage[utf8]{inputenc}
%\usepackage[automark]{scrpage2}
\usepackage{amsmath}
\usepackage{amsfonts}
\usepackage{amssymb}
\usepackage{amsthm}
\usepackage{dsfont}
\usepackage{tabularx}
\usepackage{fancyhdr}
\usepackage{graphicx}
\usepackage{subfigure}
%\usepackage{sidecap}
\usepackage{lscape}
%\usepackage{floatflt}
\usepackage{geometry}
%\usepackage{pdfpages}
\usepackage{wasysym}
\usepackage{cite}
\usepackage[german]{egplot}
\usepackage{color}
%\usepackage{epsfig}
%\usepackage{psfig}
\usepackage{a4wide}
\usepackage{totpages}
\usepackage{latexsym}
\usepackage{keyval}
\usepackage{ifthen}
\usepackage{moreverb}
\usepackage{gnuplottex}
\usepackage{enumerate}
%\usepackage{atbegshi}
\usepackage{listings}
\usepackage{pst-circ}
\usepackage{array}
\usepackage{mhchem}
\usepackage{lipsum}
\usepackage{makeidx}
\usepackage{tikz}\usetikzlibrary{shapes,arrows,automata,positioning}
\usepackage{longtable}
%\usepackage{natbib}
\usepackage{multirow}
\usepackage{caption}
\usepackage{remreset}
\usepackage[colorlinks=true,linkcolor=blue,citecolor=mauve]{hyperref}
%\usepackage{breakurl}
\usepackage[sc]{mathpazo}
%\usepackage{breakurl}
% Palatino needs more leading (space between lines)
\linespread{1.05}             
\usepackage[T1]{fontenc}      
\usepackage{bm}
\usepackage{graphicx}
\usepackage{fancyhdr}
\usepackage{ulem}
\usepackage{titlesec, blindtext, color}
\usepackage{empheq}
\usepackage{fancyref}
\usepackage{sectsty}
\usepackage{rotating}
\usepackage{makeidx}
\usepackage{xcolor}
\usepackage{transparent}
%\usepackage{setspace}
%[onehalfspacing]


%%%%%%%%%%%%%%%%%%%%%%%%%%%%%%%%%%%%%%%% COLORS %%%%%%%%%%%%%%%%%%%%%%%%%%%%%%%%%%%%%%%%
\definecolor{dblue}{HTML}{003399}
\definecolor{top}{HTML}{096B8F}
\definecolor{code}{HTML}{7C0E78}
\definecolor{dgray}{gray}{0.80}
\definecolor{dkgreen}{rgb}{0,0.6,0}
\definecolor{gray}{rgb}{0.5,0.5,0.5}
\definecolor{mauve}{rgb}{0.58,0,0.82}
\definecolor{dkgray}{gray}{0.33}

%%%%%%%%%%%%%%%%%%%%%%%%%%%%%%%%%%%%%%%% CMD %%%%%%%%%%%%%%%%%%%%%%%%%%%%%%%%%%%%%%%%
\newcommand{\cond}{\mathrm{cond}\,}
\newcommand{\supp}{\mathrm{supp}\,}
\newcommand{\myspan}{\mathrm{span}}
\newcommand{\ox}{\overline{x}}
\newcommand{\oy}{\overline{y}}
\newcommand{\ou}{\overline{u}}
\newcommand{\of}{\overline{f}}
\newcommand{\oa}{\overline{a}}
\newcommand{\ob}{\overline{b}}
\newcommand{\os}{\overline{s}}
\newcommand{\oc}{\overline{c}}
\newcommand{\mysum}{\sum\limits}
\newcommand{\myprod}{\prod\limits}
\newcommand{\infint}{\int_{-\infty}^\infty}
\newcommand{\mysim}{\quad\sim\quad}
\newcommand{\strich}{\sideset{}{'}}
\newcommand{\ggT}{\mathrm{ggT}}
\newcommand{\diag}{\mathrm{diag}\,}
\newcommand{\astleq}{\overset{(\ast)}{\leq}}
\newcommand{\dist}{\mathrm{dist}}
\newcommand{\D}{\mathrm{D}}\def\d{\;\mathrm{d}}

\DeclareMathOperator{\Res}{Res}
\DeclareMathOperator{\ess}{ess\,sup}

\let\Re\relax\let\Im\relax
\DeclareMathOperator{\Re}{Re}
\DeclareMathOperator{\Im}{Im}

\usepackage{esint}
\let\oint\ointctrclockwise
\let\phi\varphi
\let\epsilon\varepsilon

\newtheorem{thm}{Theorem}[chapter]
\newtheorem{prop}[thm]{Proposition}
\newtheorem{satz}[thm]{Satz}
\newtheorem*{satzn}{Satz}
\newtheorem{lem}[thm]{Lemma}
\newtheorem{cor}[thm]{Korollar}
\newtheorem{df}[thm]{Definition}
\theoremstyle{definition}
\newtheorem*{bspe}{Beispiele}
\newtheorem*{bsp}{Beispiel}
\newtheorem*{rem}{Bemerkung}
\newtheorem*{rems}{Bemerkungen}
\numberwithin{equation}{chapter}

\geometry{top=2.0cm, bottom=2cm, left=3cm, right=3cm}

\title{\fontsize{35pt}{60pt}\selectfont\color{dblue} Funktionenräume \\ \ \\ \Large \textbf{Dozent: } Prof. Dr. M. Griesemer}
\author{\color{dkgray}\textbf{Vorlesungsmitschrieb\footnote{Für Hinweise bezüglich Inhalt oder Form via eMail (\href{mailto:uni@robinlang.net}{uni@robinlang.net}) bin ich dankbar.
}} \ \\ \ \\\small{\color{dkgray}Stand \today}}
\date{\Large\color{dkgray}\textbf{Universität Stuttgart, Sommersemester 2015}}

\begin{document}
\maketitle
{\hypersetup{hidelinks}
\tableofcontents
}
\newpage

\section*{Motivation}
\begin{enumerate}[\bf 1)]
\item Elektron im Feld statischer Kerne. Suche $\phi\in L^2(\mathbb{R}^3)$ und $E\in\mathbb{R}$ mit
\begin{equation}
\left(-\Delta_x-\sum_{i=1}^N \frac{z_i}{|x-R_i|}\right)\phi = E\phi,
\end{equation}
wobei $z_i\in\mathbb{N}$ und $R_i\in\mathbb{R}^3$ für $i=1,\ldots,N$ ist. Die Lösungen sind im Allgemeinen nicht in $C^2(\mathbb{R}^3)$ sondern in $H^2(\mathbb{R}^3)$, d.h.
\begin{equation}
-\Delta\phi = -\sum_{i=1}^N \frac{\partial^2}{\partial_{x_i}^2}\phi
\end{equation}
ist im Sinn \textbf{schwacher Ableitungen} zu verstehen.
\item Elektrostatik: Das Potential $\Phi$ zur Ladungsverteilung $\rho\in L^1(\Omega)$, für $\Omega\subset\mathbb{R}^3$, umgeben von einem Leiter $\Omega^c$, ist bestimmt durch das Randwertproblem (RWP)
\begin{align}
\left.
\begin{array}{r l r}
-\Delta\Phi &= 4\pi\rho &\mathrm{in}\;\Omega, \\
\Phi &= 0 &\mathrm{auf}\;\partial\Omega.
\end{array}
\right\}
\end{align}
Die klassische $C^2$-Lösung minimiert das Funktional
\begin{equation}
\int_\Omega \left(|\nabla\Phi|^2 -8\pi\rho\Phi\right)\d x
\end{equation}
bezüglich allen Funktionen $\Phi$ aus $\{\Phi\in C^2(\Omega)\mid \Phi=0\text{ auf }\partial\Omega\}$, sofern sie existiert. Auch wenn der Minimierer existiert, so ist es doch einfacher, die Existenz zuerst im \textbf{Sobolev-Raum} $\mathring H^{1,2}(\Omega)$ nachzuweisen.\\
\textbf{Frage:} Wie regulär sind Funktionen aus $H^2(\mathbb{R}^3)$, $\mathring H^{1,2}(\Omega)$, etc?
\end{enumerate}

\chapter{Vorbereitung}
\begin{enumerate}[\quad\color{dblue}$\blacktriangleright$]
\item Ein \textbf{Gebiet} $\Omega\subset\mathbb{R}^n$ ist offen und zusammenhängend und $\overline{\Omega}$ ist der \textbf{Abschluss} von $\Omega$ in $\mathbb{R}^n$.
\item $G\subset\subset\Omega$ bedeutet, dass $\overline{G}\subset\Omega$ \textbf{kompakt} ist und somit $\dist(\overline{G},\Omega^c)>0$ gilt.
\item Für $u:\Omega\rightarrow\mathbb{C}$ und $\Omega\subset\mathbb{R}^n$ ist der \textbf{Träger} von $u$ definiert durch
\begin{equation}
\supp u:= \overline{\{x\in\Omega\mid u(x)\neq 0\}}\subset\mathbb{R^n}.
\end{equation}
\end{enumerate}
\subsection*{Multiindices}
Seien $\alpha,\beta\in\mathbb{N}^n$, $x=(x_1,\ldots,x_n)\in\mathbb{R}^n$ und $y\in\mathbb{R}^n$. Wir verwenden folgende Notationen:
\begin{enumerate}[\quad\color{dblue}$\blacktriangleright$]
\item $\alpha\leq\beta\quad\Leftrightarrow\quad \alpha_i\leq\beta_i$ für alle $i=1,\ldots,n$
\item $|\alpha|=\alpha_1+\cdots+\alpha_n$ und 
\item $\alpha! = \alpha_1\cdots\alpha_n$
\item $x^\alpha= x_1^{\alpha_1}\cdots x_n^{\alpha_n}$
\item $\partial^\alpha = \frac{\partial^{|\alpha|}}{\partial_{x_1}^{\alpha_1}\cdots\partial_{x_n}^{\alpha_n}}$
\item $\binom{\alpha}{\beta}=\prod_{i=1}^n \binom{\alpha_i}{\beta_i}=\prod_{i=1}^n \frac{\alpha_i!}{\beta_i!(\alpha_i-\beta_i)!}= \frac{\alpha !}{\beta!(\alpha-\beta
)!}$ für $\alpha\geq\beta$
\end{enumerate}
Damit lässt sich nun der Binomische Lehrsatz verallgemeinern:
\begin{align}
(x+y)^\alpha &= \sum_{\beta\leq\alpha} \binom{\alpha}{\beta} x^\beta y^{\alpha-\beta}, \\
\partial^\alpha (fg) &= \sum_{\beta\leq\alpha} \binom{\alpha}{\beta} (\partial^\beta f)(\partial^{\alpha-\beta}g)
\end{align}

\subsection*{Funktionenräume}
Für $\Omega\subset\mathbb{R}^n$ offen und $m\in\mathbb{N}_0$ setzen wir
\begin{enumerate}[\quad\color{dblue}$\blacktriangleright$]
\item $C^m(\Omega):=\{u:\Omega\rightarrow\mathbb{C}\mid u\text{ hat stetige partielle Ableitungen }\partial^\alpha u\text{ bis zur Ordnung }|\alpha|=m\}$
\item $C(\Omega):= C^0(\Omega):=\{u:\Omega\rightarrow\mathbb{C}\mid u\text{ ist stetig}\}$
\item $C_0^m(\Omega):=\{u\in C^m(\Omega)\mid \supp u\subset\subset\Omega\}$
\item $C^\infty(\Omega):=\bigcap_{m\geq 0} C^m(\Omega)$
\item $C_0^\infty(\Omega):=\bigcap_{m\geq 0} C_0^m(\Omega)=C^\infty(\Omega)\cap C_0(\Omega)$
\end{enumerate}
Seien $\Omega\subset\mathbb{R}^n$ (Lebesgue-)messbar und $p\geq 1$. $L^p(\Omega)$ besteht aus Äquivalenzklassen messbarer Funktionen $u:\Omega\rightarrow\mathbb{C}$ mit
\begin{equation}
\int_\Omega |u(x)|^p\d x<\infty\quad
\end{equation}
falls $1\leq p<\infty$ und
\begin{equation}
\underset{x\in\Omega}{\ess} |u(x)|:= \inf\{\alpha\geq 0\mid  |u(x)|\leq\alpha\;\mathrm{f."u.}\}<\infty
\end{equation}
falls $p=\infty$. Zwei Funktionen $u,v$ heißen \textbf{äquivalent} genau dann wenn
\begin{equation}
u\propto v\quad\Leftrightarrow\quad u(x)=v(x)\quad\text{f."u. in }\Omega.
\end{equation}
$L^p$ versehen mit den Normen
\begin{align}
\|u\|_p&:=\left(\int_\Omega |u(x)|^p\d x\right)^{1/p}\quad (1\leq p<\infty), \\
\|u\|_\infty:=\ess |u(x)|\quad(p=\infty)
\end{align}
ist ein \textbf{Banachraum}. Es gilt die \textbf{Höldersche Ungleichung}:
\begin{satzn}
Seien $f\in L^p(\Omega)$, $g\in L^q(\Omega)$ und $1\leq p,q\leq\infty$ mit $1/p+1/q=1$, dann ist $fg\in L^1(\Omega)$ und es gilt
\begin{equation}
\|fg\|_1\leq \|f\|_p\|g\|_q.
\end{equation}
\end{satzn}

\begin{thm}\label{thm1}
Ist $\Omega\subset\mathbb{R}^n$ offen und $1\leq p<\infty$, dann ist $C_0(\Omega)$ dicht in $L^p(\Omega)$.
\end{thm}

\begin{satz}\label{satz2}
Sei $u\in L^p(\mathbb{R}^n)$, $1\leq p<\infty$ und $u_h(x):= u(x-h)$. Dann gilt $\|u_h-u\|_p\rightarrow 0$ für $h\rightarrow 0$.
\end{satz}
\begin{proof}
Sei $\epsilon >0$ und wähle (siehe Theorem \ref{thm1}) $\phi\in C_0^\infty(\mathbb{R}^n)$ mit $\|u-\phi\|_p<\epsilon/3$. Dann gilt
\begin{align}
\|u_h-u\|_p &\leq \|\phi_h-\phi\|_p+ \underbrace{\|u_h-\phi_h\|_p}_{=\|u-\phi\|_h} +\|u-\phi\|_p \\
&< \|\phi_h-\phi\|_p +\frac{2}{3}\epsilon <\epsilon
\end{align}
für $|h|$ klein genug, da $\supp\phi$ kompakt und somit $\phi$ gleichmäßig stetig ist.
\end{proof}

\subsection*{Faltung und Glättung}
Seien $f,g:\mathbb{R}^n\rightarrow\mathbb{C}$ messbar, $x\in\mathbb{R}^n$ und sei $y\mapsto f(x-y)g(y)$ integrierbar, dann ist
\begin{equation}
(f\ast g)(x):= \int_{\mathbb{R}^n} f(x-y)g(y)\d y=(g\ast f)(x)
\end{equation}
die \textbf{Faltung} von $f$ mit $g$.

\begin{satz}\label{satz3}
Sei $1\leq p\leq \infty$. Falls $f\in L^p(\mathbb{R}^n)$ und $g\in L^1(\mathbb{R}^n)$, dann ist auch $f\ast g\in L^p(\mathbb{R}^n)$ und es gilt
\begin{equation}
\|f\ast g\|_p\leq \|f\|_p\|g\|_1.
\end{equation}
\end{satz}
\begin{proof}
Der Fall $p=1,\infty$ verbleibt als Übung. Sei also $1<p<\infty$ und $q$ so, dass $1/p+1/q=1$ gilt. Dann ist
\begin{align}
|f\ast g(x)| &\leq \int_{\mathbb{R}^n} |f(x-y)|\cdot |g(y)|^{1/p}\cdot |g(y)|^{1/q}\d y \\
&\leq \|g\|_1^{1/q}
\left(
\int_{\mathbb{R}^n} |f(x-y)|^p\cdot  |g(y)\d y
\right)^{1/p}
\end{align}
und somit
\begin{align}
\int_{\mathbb{R}^n}|f\ast g(x)|^p\d x 
&\leq \int_{\mathbb{R}^n} \left(\int_{\mathbb{R}^n} |f(x-y)|^p\cdot |g(y)|\d y\right)\cdot \|g\|_1^{p/q} \d x \\
&= \|f\|_p^p\cdot \|g\|_1^{1+p/q} <\infty.
\end{align}
Durch Wurzelziehen folgt die Behauptung.
\end{proof}

\begin{thm}[Young'sche Ungleichung]\label{thm4}
Seien $1\leq p,q\leq\infty$ und $1/p+1/q=1+1/r$. Falls $f\in L^p(\mathbb{R}^n)$ und $g\in L^q(\mathbb{R}^n)$, dann ist $f\ast g\in L^r\mathbb{R}^n$ und es gilt
\begin{equation}
\|f\ast g\|_r\leq \|f\|_p\|g\|_q.
\end{equation}
\end{thm}
\begin{proof}
Siehe \cite{AF}.
\end{proof}

Sei im Folgenden
\begin{equation}
L_\mathrm{loc}^p(\Omega):=\{u:\Omega\rightarrow\mathbb{C}\mid u\text{ ist messbar mit }u\in L^p(K)\text{ für beliebige }K\subset\subset\Omega\}.
\end{equation}

\begin{lem}\label{lem5}
Sei $J\in C_0^\infty(\mathbb{R}^n)$, dann gilt für $u\in L_\mathrm{loc}^1(\mathbb{R}^n)$
\begin{enumerate}[\bf (a)]
\item $J\ast u\in C^\infty(\mathbb{R}^n)$ und $\partial^\alpha(J\ast u)=(\partial^\alpha J)\ast u$ für $\alpha\in\mathbb{N}_0^n$.
\item Falls $\supp J\subset \overline{B_\epsilon(0)}$, dann gilt $\supp (J\ast u)\subset\supp (u)_\epsilon$\footnote{Dies ist die Menge aller $x$ mit $\dist(x,\supp u) <\epsilon$ gemeint.}
\end{enumerate}
\end{lem}
\begin{proof}
\begin{enumerate}[\bf (a)]
\item Skizze: (1) $J\ast u\in C(\mathbb{R}^n)$, (2) $\partial_{x_i}(J\ast u)=(\partial_{x_i} J)\ast u$ mit Satz von Lebesgue, (3) Induktion.
\end{enumerate}
\end{proof}

\begin{bspe}
\begin{enumerate}[\bf (a)]
\item Für die Funktion $J$ gegeben durch
\begin{equation}
J(x):=
\begin{cases}
\exp\left(-\frac{1}{1-|x|^2}\right) &\mathrm{falls}\; |x|<1,\\
0 &\mathrm{falls}\; |x|\geq 1
\end{cases}
\end{equation}
gilt $J\in C_0^\infty(\mathbb{R}^n)$.
\item Sei $0\leq J\in C_0^\infty(\mathbb{R}^n)$ mit $\supp J\subset \{|x|\leq 1\}$ und $\int J(x)\d x=1$. Für $\epsilon>0$ setzen wir
\begin{equation}
J_\epsilon(x):=\epsilon^{-n} J(x/\epsilon).
\end{equation}
Dann gilt
\begin{enumerate}[(i)]
\item $J_\epsilon\in C_0^\infty(\mathbb{R}^n)$, $J_\epsilon\geq 0$ und $\supp J_\epsilon\subset\overline{B_\epsilon(0)}$,
\item $\int J_\epsilon(x)\d x =1$.
\end{enumerate}
\end{enumerate}
\end{bspe}

\begin{lem}\label{lem6}
Sei $u\in L_\mathrm{loc}^1(\mathbb{R}^n)$ stetig in der offenen Menge $\Omega\subset\mathbb{R}^n$. Dann gilt für jede kompakte Menge $K\subset\Omega$
\begin{equation}
\sup\limits_{x\in K} |J_\epsilon\ast u(x)-u(x)|\rightarrow 0\qquad (\epsilon\rightarrow 0^+).
\end{equation}
\end{lem}

\begin{proof}
Es ist
\begin{align}
J_\epsilon\ast u(x)-u(x)
&= \int J_\epsilon(x-y)u(y)\d y - \underbrace{\int J_\epsilon(x-y)\d y}_{=1} u(x) \\
&= \int J_\epsilon(x-y)\big( u(y)-u(x)\big)\d y
\end{align}
und somit
\begin{align}
\left |J_\epsilon\ast u(x)-u(x)\right| 
&\leq \int_{|x-y|\leq\epsilon} J_\epsilon(x-y) |u(y)-u(x)| \d y \\
&\leq\sup\limits_{y:|y-x|\leq \epsilon} |u(y)-u(x)|.
\end{align}
Sei $\epsilon <\epsilon_0:=\dist(K,\Omega^c)$. Dann ist
\begin{align}
\sup\limits_{x\in K} \left |J_\epsilon\ast u(x)-u(x)\right|  
\leq \sup\limits_{x\in K_\epsilon,\; y\in K_\epsilon,\;|x-y|\leq\epsilon} \left |J_\epsilon\ast u(x)-u(x)\right|   \rightarrow0 \qquad (\epsilon\rightarrow 0^+),
\end{align}
da $u$ auf $K_\epsilon$ gleichmäßig stetig ist. 
\end{proof}

\begin{thm}\label{thm7}
Sei $\Omega\subset\mathbb{R}^n$ offen,  $1\leq p <\infty$ und $u\in L^p(\Omega)$. Dann gilt
\begin{enumerate}[\bf (a)]
\item $J_\epsilon \ast u \in C^\infty(\Omega)\cap L^p(\Omega)$,
\item $\|J_\epsilon\ast u\|_{p,\Omega}\leq \|u\|_{p,\Omega}$,
\item $\|J_\epsilon\ast u -u\|_{p,\Omega}\rightarrow 0$ für $\epsilon\rightarrow 0^+$,
\end{enumerate}
wobei $J_\epsilon\ast u(x)=\int_\Omega J_\epsilon(x-y)g(y)\d y$ ist, d.h. setze $u$ in $\mathbb{R}^n\backslash\Omega$ durch $u(x)=0$ fort.
\end{thm}

\begin{proof}
\begin{enumerate}[\bf (a)]
\item Aus $u\in L^p(\Omega)$ folgt $u\in L_{\mathrm{loc}}^1(\mathbb{R}^n)$ (siehe Blatt 1). Also ist nach Lemma \ref{lem5} und Satz \ref{satz3} $J_\epsilon\ast u\in C^\infty(\Omega)\cap L^p(\mathbb{R}^n)$. 
\item Aus Satz \ref{satz3} folgt weiter, dass
\begin{equation}
\|J_\epsilon\ast u\|_{p,\Omega} 
\leq \|J_\epsilon\ast u\|{p,\mathbb{R}^n}
\leq \underbrace{\|J_\epsilon\|_1}_{=1}\underbrace{\|u\|_{p,\mathbb{R}^n}}_{=\|u\|{p,\Omega}}.
\end{equation}
\item Nach Theorem \ref{thm1} existiert ein $\Phi\in C_0(\Omega)$ mit 
\begin{equation}
|u-\Phi\|_p <\delta/3 \quad\mathrm{f"ur}\;\delta >0.
\end{equation}
 Nach Lemma \ref{lem6} konvergiert dann
\begin{equation}
|J_\epsilon\ast \Phi -\Phi|\rightarrow 0\qquad (\epsilon\rightarrow 0^+)
\end{equation}
gleichmäßig auf $K:=\supp (\Phi)_1=\{x\mid \dist(x,\supp\Phi)\leq 1\}$. Also ist
\begin{align}
\|J_\epsilon\ast \Phi-\Phi\|_p^p 
&= \int \big| J_\epsilon\ast\Phi(x)-\Phi(x) \big|^p\d x \\
&\leq \sup\limits_{x\in K} \big| J_\epsilon\ast\Phi(x)-\Phi(x) \big|^p \int_K 1\d x\rightarrow 0 \qquad (\epsilon\rightarrow 0^+).
\end{align}
Somit existiert ein $\epsilon_0 >0$ so, dass $\|J_\epsilon\ast \Phi-\Phi\|_p <\delta/3$ für $\epsilon<\epsilon_0$ und es folgt
\begin{align}
\|J_\epsilon\ast u-u\|_p 
&\leq \|J_\epsilon\ast (u-\Phi)\|_p +\|J_\epsilon\ast \Phi-\Phi\|_p +\|\Phi-u\|_p \\
&\leq \|J_\epsilon\|_1\cdot \|u-\Phi\| +\frac{2}{3}\delta <\delta.
\end{align}
\end{enumerate}
\end{proof}


\begin{satz}\label{satz8}
Sei $\Omega\subset\mathbb{R}^n$ offen und $1\leq p<\infty$. Dann ist $C_0^\infty(\Omega)$ dicht in $L^p(\Omega)$.
\end{satz}

\begin{proof}
Nach Theorem \ref{thm1} ist $C_0(\Omega)\subset L^p(\Omega)$ dicht. Sei also $u\in L^p(\Omega)$, $\delta>0$ und $\Phi\in C_0(\Omega)$ mit $\|u-\Phi\|_p<\delta/2$. Nach Lemma \ref{lem5} ist dann $J_\epsilon\ast\Phi\in C_0^\infty(\Omega)$ falls $\epsilon<\dist(\supp\Phi,\Omega^c)$ und
\begin{align}
\|J_\epsilon\ast\Phi-\Phi\|_p <\frac{\delta}{2}
\end{align}
für $\epsilon$ klein genug (Theorem \ref{thm7}(c)). Also ist
\begin{align}
\|J_\epsilon\ast\Phi-u\|_p 
\leq \|J_\epsilon\ast\Phi-\Phi\|_p+\|\Phi-u\|_p 
< \frac{\delta}{2}+\frac{\delta}{2}=\delta
\end{align}
für $\epsilon$ klein genug.
\end{proof}

\begin{satz}\label{satz9}
Sei $\Omega\subset\mathbb{R}^n$ offen und $u\in L_{\mathrm{loc}}^1(\Omega)$. Falls
\begin{equation}
\int_\Omega u\phi\d x=0\quad\text{f"ur alle }\phi\in C_0^\infty(\Omega),
\end{equation}
dann ist $u(x)=0$ fast überall in $\Omega$.
\end{satz}

\begin{proof}
Für $n\in\mathbb{N}$ sei
\begin{equation}
K_n=\{x\in\Omega\mid |x|\leq n\text{ und }\dist(x,\Omega^c)\geq 1/n\}.
\end{equation}
Also ist $K_n\subset\Omega$ kompakt, $K_n\subset K_{n+1}$ und $\bigcup_{n\geq 1} K_n=\Omega$. Weiter ist
\begin{equation}
\dist(K_n,K_{2n}^c)\geq \frac{1}{2n}.
\end{equation}
Sei
\begin{equation}
u_n(x)=
\begin{cases}
u(x)\chi_{K_n}(x), &\mathrm{falls}\; x\in\Omega ,\\
0, &\mathrm{falls}\; x\notin\Omega.
\end{cases}
\end{equation}
Dann ist$u_n\in L^1(\Omega)$ und für $x\in K_n$ und $\epsilon\leq 1/(2n)$ gilt
\begin{align}
J_\epsilon\ast u_{2n}(x) 
= \int_{|x-y|\leq\frac{1}{2n}} J_\epsilon(x-y)u_{2n}(y)\d y 
= \int_\Omega J_\epsilon(x-y)u(y)\d y 
=0,
\end{align}
da $y\mapsto J_\epsilon(x-y)$ in $C_0^\infty(\Omega)$ liegt. Es folgt $\chi_{K_n}(J_\epsilon\ast u_{2n})\equiv 0$, wobei
\begin{equation}
J_\epsilon\ast u_{2n} \rightarrow u_{2n}\quad\mathrm{in}\; L^1(\Omega). 
\end{equation}
Also gilt
\begin{align}
\|\chi_{K_n}u\|_1 
= \|\chi_{K_n}u_{2n}\|_1
= \lim\limits_{\epsilon\rightarrow 0^+} \|\chi_{K_n}(J_\epsilon\ast u_{2n})\|_1 =0,
\end{align}
d.h. $u(x)=0$ fast überall in $K_n$ und somit auch
\begin{equation}
u(x)=0\quad\mathrm{fast\;"uberall\; in} \bigcup\limits_{n\geq 1} K_n =\Omega.
\end{equation}
\end{proof}




\chapter{Sobolev-Räume}
\section*{Schwache Ableitung}
Sei $\Omega\subset\mathbb{R}^n$ offen und $u\in C^k(\Omega)$ für $k\in\mathbb{N}$. Dann gilt für alle $\alpha\in\mathbb{N}_0^n$ mit $|\alpha|\leq k$ und für alle $\phi\in C_0^\infty(\Omega)$ die Identität
\begin{equation}
\int_\Omega u\partial^\alpha\phi\d x= (-1)^{|\alpha|}\int_\Omega (\partial^\alpha u)\phi \d x.
\end{equation}
Das motiviert folgende Definition:

\begin{df}
Sei $\alpha\in\mathbb{N}_0^n$ und $u,b\in L_{\mathrm{loc}}^1(\Omega)$, $\Omega\subset\mathbb{R}^n$ offen mit
\begin{equation}
\int_\Omega u\partial^\alpha\phi\d x= (-1)^{|\alpha|}\int_\Omega v\phi \d x
\end{equation}
für alle $\phi\in C_0^\infty(\Omega)$. Dann heißt $v$ \textbf{schwache} $\boldsymbol{\alpha}$\textbf{-Ableitung} von $u$ und man schreibt $v=\partial^\alpha u$.
\end{df}

\begin{rems}
\begin{enumerate}[\bf 1)]
\item Die schwache $\alpha$-Ableitung ist eindeutig, falls sie exisitiert: Sind $v,\tilde{v}$ schwache $\alpha$-Ableitungen von $u$, dann gilt
\begin{equation}
\int_\Omega (v-\tilde{v})\phi\d x =0\quad\mathrm{f"ur\; alle\;}\phi\in C_0^\infty(\Omega)
\end{equation}
und somit $v=\tilde{v}$ f.ü. in $\Omega$. D.h. $v=\tilde{v}$ in  $L_{\mathrm{loc}}^1(\Omega)$. 
\item Falls $u\in C^k(\Omega)$, dann ist $\partial^\alpha u$ für $|\alpha|\leq k$ die klassische $\alpha$-Ableitung von $u$.
\item Es ist möglich, dass $\partial^\alpha u$ existiert aber $\partial^\beta u$ für ein $\beta\leq \alpha$ nicht existiert.
\end{enumerate}
\end{rems}

\begin{bspe}
\begin{enumerate}[\bf 1)]
\item Sei $u(x)=x\cdot \chi_{\{x\geq 0\}}(x)$. Dann ist $u'=\Theta$ die \textit{Heaviside Funktion} aber $\Theta$ hat keine schwache Ableitung ($\Theta '=\delta$ im Distributionssinn).
\begin{proof}
\begin{align}
\int\limits_{-\infty}^\infty u\phi' \d x=\int\limits_0^\infty x\phi'(x)\d x= x\phi(x)\Big|_0^\infty -\int\limits_0^\infty \phi(x)\d x =-\int\limits_{-\infty}^\infty \Theta(x)\phi(x)\d x
\end{align}
\end{proof}
\item Sei $u(x,y)=\Theta(x)$ für $(x,y)\in\mathbb{R}^2$. Sei $\alpha=(1,1)$, dann gilt $\partial^\alpha u=0$ aber $\partial_x u$ existiert nicht.
\begin{proof}
\begin{align}
\int u\partial^\alpha \phi \d x\d y 
&= \int\limits_{-\infty}^\infty\d x \int\limits_{-\infty}^\infty\d y\Theta(x) \frac{\partial}{\partial y}\left(\frac{\partial\phi}{\partial x}\right)\\
&= \int\limits_{-\infty}^\infty \d x\Theta(x) \underbrace{\int\limits_{-\infty}^\infty\d y \frac{\partial}{\partial y}\left(\frac{\partial\phi}{\partial x}\right)}_{=0} =0.
\end{align}
Also ist $\partial^\alpha u =0$.
\end{proof}
\item Für $\kappa<n-1$ gilt
\begin{align}
\partial_i |x|^{-\kappa} =-\kappa \frac{x_i}{|x|^{\kappa +2}}.
\end{align}
\end{enumerate}
\end{bspe}

\subsection{Sobolevräume}
Sei $\Omega\subset\mathbb{R}^n$ offen, $1\leq p\leq\infty$ und $m\in\mathbb{N}$. Dann ist
\begin{equation}
W^{m,p}(\Omega):=\{u\in L^p(\Omega)\mid \partial^\alpha u\in L^p(\Omega)\text{ f"ur }\alpha:\;|\alpha|\leq m\}
\end{equation}
mit
\begin{align}
\|u\|_{m,p}&:=\left(\sum\limits_{|\alpha|\leq m}\|\partial^\alpha u\|_p^p\right)^{1/p}\qquad (1\leq p<\infty) \\
\|u\|_{m,\infty} &:= \max\limits_{|\alpha|\leq m} \|\partial^\alpha u\|_\infty \qquad (p=\infty)
\end{align}
ist ein normierter Vektorraum.

\begin{thm}\label{thm2_1}
Für $1\leq p\leq\infty$ ist $W^{m,p}(\Omega)$ ein Banachraum.
\end{thm}

\begin{proof}
Sei $(u_k)_{k=1}^\infty$ eine Cauchy-Folge in $W^{m,p}(\Omega)$, dann ist $(\partial^\alpha u_k)$ für jedes $|\alpha|\leq m$ eine Cauchy-Folge in $L^p(\Omega)$ (dieser ist vollständig). Also existiert ein $u_\alpha$ mit
\begin{equation}
\partial^\alpha u_k \rightarrow u_\alpha\quad\mathrm{in}\; L^p(\Omega)
\end{equation}
für alle $\alpha$ mit $|\alpha|\leq m$. Sei $u=u_{\alpha =0}$. Zu zeigen ist $u_\alpha=\partial^\alpha u$ für alle $\alpha$ mit $|\alpha|\leq m$. \\\\
Sei $\phi\in C_0^\infty(\Omega)$, dann
\begin{align}
\int_\Omega u\partial^\alpha u\d x 
&\overset{(\ast)}{=} \lim\limits_{k\rightarrow\infty} \int_\Omega u_k\partial^\alpha\d x \\
&= \lim\limits_{k\rightarrow\infty} (-1)^{|\alpha|}\int_\Omega \partial^\alpha u_k\phi\d x \\
&= (-1)^{|\alpha|} \int_\Omega u_\alpha\phi \d y .
\end{align}
Also ist $u_\alpha =\partial^\alpha u$ und somit $\partial^\alpha u_k\rightarrow\partial^\alpha u$ in $L^p(\Omega)$ für alle $\alpha$ mit $|\alpha|\leq m$ und somit $\|u_k-u\|_{m,p}\rightarrow 0$ für $k\rightarrow\infty$.
Zu $(\ast)$:
\begin{align}
\left| \int_\Omega\left( u\partial^\alpha\phi-u_k\partial^\alpha\phi \right)\d x \right| 
\leq \|u-u_k\|_p \|\partial^\alpha \phi\|_q\rightarrow 0 \qquad (k\rightarrow\infty),
\end{align}
wobei $1/q+1/q=1$.
\end{proof}

\begin{bspe}
\begin{enumerate}[1)]
\item Sei $\Omega = B_R(0)\subset \mathbb R^n$ und $$u(x)=|x|^{-\alpha} \quad \alpha <n$$

Dann ist $u\in L_{\text{loc}}^1(\Omega)$ und 
$$\nabla u(x)= - \alpha \frac{x}{|x|^{\alpha+2}} \quad \alpha < n-1$$
(Blatt 1). Es gilt
\begin{align*}
\int_{|x|<R} |\nabla u|^p \, \mathrm dx &= \alpha \int_{|x|<R} \frac{1}{|x|^{(\alpha+1)p}} \, \mathrm d^n x\\
&= \begin{cases}
\alpha \omega_n \frac{R^{n-(\alpha+1)p}}{n-(\alpha+1)p} &\ \quad \alpha < \frac{n}{p} -1\\
\infty &\ \quad \alpha \ge \frac{n}{p} -1
\end{cases}
\end{align*}
wobei $\omega_n = \int_{|x|=1}$ der Flächeninhalt der Einheitssphäre in $\mathbb R^n$ ist. Im Fall $\alpha < \frac{n}{p}-1$ folgt $u\in W^{1,p}(B_R(0))$ (dann $u\in L^p(B_R(0))$ Übung). Es gilt auch $u\in W^{1,p}(B_R(0))$. Dann folgt $\alpha < \frac{n}{p}-1$ (Übung). 

Also
$$
u \in W^{1,p}(B_R(0)) \iff \alpha < \frac{n}{p}-1
$$
\item Sei $(d_k)_{k\ge 1}$ dicht in $B_1(0)$ und 
\begin{equation}\label{bsp1-eq1}
u(x) = \sum_{k\ge 1} 2^{-k} |x-a_k|^{-\alpha}
\end{equation}
Dann ist $u\in W^{1,p}(B_1(0))$ genau dann wenn $\alpha < \frac{n}{p}-1$.
\begin{proof}
Falls $\alpha < \frac{n}{p}-1$, dann ist $u(x)=2^{-k} |x-a_k|^{-\alpha}$ in $W^{1,p}(B_1(0))$ und $\| u_k\|_{1,p} \le i^k C_{\alpha, p,n}$, also ist die Reihe $\sum_{k\ge 1} \| u_k \|_{W^{1,p}(B_1(0))} < \infty$, d.h. \eqref{bsp1-eq1} ist absolut konvergent in $W^{1,p}(B_1(0))$ und somit $u\in W^{1,p}(B_1(0))$ denn $W^{1,p}$ ist vollständig. Übung: $u\in W^{1,p}(B_1(0))\implies \alpha < \frac{n}{p}-1$. 
\end{proof}
\end{enumerate}
\end{bspe}

\begin{prop}
Falls $n>p$ und $0<\alpha < \frac{n}{p}-1$ dann ist $u\in W^{1,p}(B_1(0))$ und trotzdem in jedem Punkt $d_k$ divergent.
\end{prop}
%\end{enumerate}
Wir wollen nun zeigen, dass $C^\infty(\Omega) \cap W^{1,p}(\Omega)$ dicht ist in $W^{1,p}(\Omega)$, $1\le p <\infty$. Dazu brauchen wir einige Vorbereitungen:

\begin{satz}
Seien $u,v\in W^{m,p}(\Omega)$, $1\le p \le \infty$ und $|\alpha |\le m$. Dann gilt
\begin{enumerate}[i)]
\item $\partial^\alpha u\in W^{m-|\alpha|, p}(\Omega)$ und $\partial^\beta(\partial^\alpha u) = \partial^\alpha (\partial^\beta u) = \partial^{\alpha + \beta} u$ falls $|\alpha|+|\beta|\le m$.
\item $\lambda u + \mu v\in W^{m,p}(\Omega)$ und $\partial^\alpha (\lambda u + \mu v)= \lambda \partial^\alpha u + \mu \partial^\alpha v$ für $\lambda, \mu \in \mathbb C$.
\item Ist $v\subset \Omega$ offen, dann ist $u\upharpoonright V\in W^{m,p}(V)$. 
\item Ist $\gamma \in C_0^\infty(\Omega)$, dann ist $\gamma u \in W^{m,p}(\Omega)$ und
$$
\partial^\alpha (\gamma u) = \sum_{\beta \le \alpha} \binom{\alpha}{\beta} (\partial^\beta \gamma) ( \partial^{\alpha-\beta}u)
$$ 
\end{enumerate}
\end{satz}
\begin{proof}
(i)-(iii) Übung (L.C. Evans).\\
(iv) Beweis der  Leibniz-Regel
\begin{align*}
\int(\gamma u) \partial_i \phi \, \mathrm dx &= \int u (\gamma \partial_i \phi) \\
&= \int u (\partial_i (\gamma \phi) - (\partial_i \gamma) \phi) \\
&= - \int (\partial_i u) (\gamma \phi) + u(\partial_i \gamma) \phi \\
&= - \int ((\partial_i u) \gamma + u\partial_i \gamma)) \phi.
\end{align*}
Per Induktion bekommt man nun die Leibnizregel für $\partial^\alpha$ (s. Evans).
Aus der Leibniz-Regel folgt $\gamma u \in W^{m,p}(\Omega)$ denn $\partial^\beta \gamma\in C_0^\infty$ und $\partial^{\alpha-\beta}u \in L^p(\Omega)$.
\end{proof}

\begin{lem}
Ist $K\subset \Omega$ kompakt, dann existiert $\phi \in C_0^\infty(\Omega)$ mit $0\le \phi \le 1$ und $\phi \equiv 1$ auf $K$ und
$$
\sup_{x\in \Omega} |\partial^\alpha \phi(x)|\le c_\alpha \delta^{-|\alpha|}
$$ 
wobei $\delta = \dist(K, \Omega^c)$ und $c_\alpha$ ist unabhängig von $K, \Omega$.
\end{lem}
\begin{proof}
Sei $X_\delta$ die charakteristische FUnktion von $K_{\delta/2}:= \{x\in \Omega|\text{dist} (x,K) \le \frac{\delta}{2}\}$. Sei $\varepsilon = \frac{\delta}{3}$ und $\phi:= J_\varepsilon * \chi_\delta$. Dann ist $0 \le \phi \le 1, \phi \in C_0^\infty(\mathbb R^n)$ und $\text{supp}(\phi) \subset \text{supp}(\chi_\delta) \subset K_{\delta/2+ \varepsilon} \subset \Omega$ nach Lemma 1.5. Außerdem gilt
$$
\partial^\alpha \phi = (\partial^\alpha J_\varepsilon) * \chi_\delta,
$$ 
wobei $\partial^\alpha J_\varepsilon(x) = \varepsilon^{-|\alpha|} (\partial_\alpha J)_\varepsilon (x)$ und somit
\begin{align*}
\|\partial^\alpha \phi\|_\infty &\le \| \partial^\alpha J_\varepsilon \|_1 \underbrace{\| \chi_\delta\|_\infty}_{=1} \\
&= \varepsilon^{-|\alpha|} \| (\partial^\alpha J)_\varepsilon\|_1 = \varepsilon^{-|\alpha|} \| \partial^\alpha J\|_1
\end{align*} 
\end{proof}

\begin{satz}[Zerlegung der Eins]
Sei $\Omega \subset \mathbb R^n$ und $\Omega = \bigcup_{U\in \mathcal O} U$ eine offene Überdeckung von $\Omega$, $U\subset \mathbb R^n$ offen. Dann existiert eine Folge $\psi_k \in C_0^\infty(\Omega), k \in \mathbb N$ mit
\begin{enumerate}[(i)]
\item $0 \le \psi_k\le 1$.
\item $\text{supp}(\psi_k) \subset U$ für ein $u\in \mathcal O$.
\item Ist $K\subset \Omega$ kompakt, dann existiert $W\supset K$ offen, $K\subset W\subset \Omega$ und $m\in \mathbb N$, so dass
$$
\sum_{k=1}^m \psi_k(x) =1 \quad x\in W
$$
(bzw. $\sum_{k\ge 1} \psi_k(x) =1$ in $\Omega$).  $(\psi_k)$ heißt eine \textbf{der offene Überdeckung $\Omega = \bigcup_{U\in \mathcal O} U$ untergeordnete, lokal endliche Zerlegung der Eins}.
\end{enumerate} 
\end{satz}
\begin{proof}
Sei $D\subset \Omega$ eine abzählbar und dicht und sei $(B(x_j, r_j))_{j\in \mathbb N}$ die Folge der abgeschlossenen Kugeln welche alle Kugeln $\overline{B(x, r)}$ mit $X\subset D, r\in \mathbb Q$ und $\overline{B(x,r)}\subset U$ für ein $U\in \mathcal O$ umfasst.


Sei $V_j= \{x||x-x_j|<\frac{r_j}{2}\} \subset B(x_j, r_j)$.  Dann existiert $\phi_j \in C_0^\infty(\Omega)$ mit
\begin{enumerate}[(i)]
\item $0\le \phi_j \le 1$
\item $\phi_j \equiv 1$ auf $V_j$
\item $\supp(\phi_j) \subset B(x_j, r_j)$
\end{enumerate}
(vgl. Lemma 4). Definiere
\begin{align*}
\psi_1&:= \phi_1 \\
\psi_2 &:= (1-\phi_1) \phi_2\\
&\vdots \\
\psi_j &:= (1-\phi_1)(1-\phi_2) \cdots (1- \phi_{i-1}) \phi_i.
\end{align*}

Dann gilt $0 \le \psi_j \le 1$, $\text{supp}(\psi_j) \subset \text{supp}(\phi_j) \subset \overline{B(x_i, r_i)}$ und
$$
\psi_1 + \psi_2 + \ldots + \psi_i = 1- \prod (1-\phi_i)
$$
(FIXME)
Da $\phi_i =1$ in $V_i$ folgt $\psi_1 + \psi_2 + \cdots + \psi_i =1$ in $V_1 \cup V_2 \cup V_3 \cup \ldots \cup V_j$. Sei $K\subset \Omega$ kompakt, dann existiert $m\in \mathbb N$ mit $K\subset \bigcup_{j=1}^m V_j =:W $ denn
\end{proof}

\begin{lem}\label{2.6}
Sei $u\in W^{m,p}(\Omega), 1 \le p <\infty$ und sei $V\subset \subset \Omega$ offen. Dann gilt
$$
\| J_\varepsilon * u - u\|_{W^{m,p}(V)} \to 0 \quad (\varepsilon \rightarrow 0 +)
$$
\end{lem}
\begin{proof}
Wir zeigen zuerst, dass
$$
\partial^\alpha (J_\varepsilon * u) = J_\varepsilon * (\partial^\alpha u)
$$
in $V$ für $|\alpha|\le m$ und $\varepsilon < \text{dist} (V, \Omega^c)$. Nach Lemma 1.5 ist $J_\varepsilon * u\in C^\infty(\Omega)$ und 
$$
\partial^\alpha (J_\varepsilon * u) = ( \partial^\alpha J_\varepsilon) * u.
$$
Sei $x\in V$ und $\varepsilon < \text{dist}(V, \Omega^c)$ Dann ist die Funktion 
$y\mapsto J_\varepsilon(x-y)$ in $C_0^\infty(\Omega)$ und somit 
\begin{align*}
(\partial^\alpha J_\varepsilon * u) (x)&= \int \partial^\alpha J_\varepsilon (x-y) u(y) \, \mathrm dy\\
&= (-1)^{|\alpha|} \int \partial_y^\alpha J_{\varepsilon} (x-y) u(y) \, \mathrm dy \\
&= \int J_\varepsilon (x-y) \partial^\alpha u(y) \, \mathrm dy \\
&= J_\varepsilon * (\partial^\alpha u) (x).
\end{align*}
$\partial^\alpha u \in L^p(V)$, $1\le p <\infty$. Also nach Theorem 1.7, $J_\varepsilon * \partial^\alpha u \to \partial^\alpha u$ in $L^p(V)$ ($\varepsilon \to 0 +$). Es folgt
\begin{align*}
\| J_\varepsilon * u - u \|_{W^{1,p}(V)} = \sum_{|\alpha \le m} \| \partial^\alpha (J_\varepsilon * u) - \partial^\alpha u \|^p_{p,V}\\
&= \sum_{|\alpha|\le m} \| J_\varepsilon * (\partial^\alpha u) - \partial^\alpha u \|_{p,V}^p \to 0 (\varepsilon 0 +),
\end{align*}
\end{proof}

\begin{thm}[Meyers, Serrin 1964]
Für $1\le p <\infty$ ist $C^\infty(\Omega) \cap W^{m,p}(\Omega)$ dicht in $W^{m,p}(\Omega)$.
\end{thm}
\begin{proof}
Für $k\in \mathbb N$ sei
$$
\Omega_k = \{x\in \Omega |\text{dist} (x, \Omega^c) >\frac{1}{k} \quad \text{ und } |x| <k\}.
$$
Dann $\Omega_1 \subset \Omega_2 \subset \cdots \subset \Omega$ und $\bigcup_{k\ge 1} \Omega_k = \Omega$ für 
$k\ge 2$. Sei $$U_k = \Omega_{k+1} \cap \overline{\Omega_{k-1}}^c = \Omega_{k+1} \setminus \overline{\Omega_{k-1}}$$
und $U_1= \Omega_1$ Dann $\Omega = \bigcup_{k=1}^\infty U_k$. Sei $(\phi_j)$ eine der offene Überdeckung $\Omega = \bigcup_{i \ge 1} U_i$ untegeordnete lokal endliche Zerlegung der Eins (Satz 5) und sei $(\psi_k)_{k\ge 1}$ wie folgt definiert. $\psi_1$ ist die Summe der $\phi_i$ mit $\supp(\phi_i) \subset U_1$. $\phi_2$ ist die Summe der $\phi_i$ mit $\text{supp}(\phi_i) \subset U_2$ aber $\text{supp}(\phi_i) \not \subset U_1$ etc.  Dann ist $\psi_k \in C_0^\infty(\Omega)$ dann $\overline U_k$ kompakt und somit ist $\psi_k$ eine endliche Summe. Außerdem $0 \le \psi_k \le 1, \sum \psi_k(x)=1$ in $\Omega$, $\text{supp}(\psi_k) \subset U_k$.  Sei $\varepsilon>0$ und $\varepsilon_k >0$ so klein, dass  
$$
\text{supp}(J_{\varepsilon_k} * ( \psi_k U)) \subset U_k
$$
und $$\| J_{\varepsilon_k}*\underbrace{(\psi_k u)}_{\in W^{m,p}}- \psi_k u\|_{W^{m,p}(\Omega_k)} < 2^{-k}\varepsilon$$
(nach Lemma 6). Definiere 
$$
\phi:= \sum_{k\ge 1} J_{\varepsilon_k} * (\psi_\varepsilon U)
$$
auf jeder kompakten Menge $K\subset \Omega$ sind nur endlich viele Summanden $\neq 0$ also $\phi \in C^\infty(\Omega)$. In $\Omega_k$ gilt
\begin{align*}
u(x) &= \sum_{j=1}^{k+1} \psi_j(x) u(x)\\
\phi(x) &= \sum_{j=1}^{k+1} J_{\varepsilon_i} * (\psi_j u) (x).
\end{align*}
Also gilt 
\begin{align*}
\| \phi - u \| _{W^{m,p}(\Omega_k)} &\le \sum_{i=1}^{k+2} \| J_{\varepsilon_i} * (\psi_i u) - \psi_i u\|_{W^{m,p}(\Omega_k)} \\
&\le \sum_{i=1}^{k+1} \varepsilon \cdot 2^{-j} < \varepsilon
\end{align*}
Mit monotoner Konvergenz folgt $\|\phi- u\|_{W^{m,p}(\Omega)}\le \varepsilon$.
\end{proof}
\section{Einbettungssätze}
\subsection{Sobolev-Ungleichungen}
\begin{bsp}
	Es gilt
	\begin{equation}
		u: x \mapsto \frac{1}{|x|^\alpha} \in W^{1,p}(B_1(0)) \iff \alpha < \frac{n}{p}-1
	\end{equation}
\end{bsp}
Dieses Beispiel zeigt, dass mit steigender Dimension $n$ FUnktionen mit "'schlimmeren`` Singularitäten immer noch in $W^{1,p}(\Omega)$ liegen können. In diesem Kapitel ist immer $p<n$ (später $p>n$, dann $W^{m,p}(\Omega) \subset C^k(\Omega)$ für $k<m-\frac{n}{p}$).

Sei $1\le p <n$. Gibt es ein $q\ge 1$ und ein $C\in \mathbb R$, so dass
\begin{equation}\label{sobolev1}
\| u\|_q \le C \| \nabla u \|_p \quad \forall u\in C_0^1(\mathbb R^n)
\end{equation}
gilt?  Falls \eqref{sobolev1} stimmt, dann gilt auch 
\begin{equation}\label{sobolev2}
\| u_\lambda\|_q \le C \| \nabla u_\lambda\|_p
\end{equation}
für alle $\lambda>0$, wobei $u_\lambda(x):= u(\lambda x)$ ist. Es gilt
\begin{itemize}
\item $\int_{\mathbb R^n} |u_\lambda(x)|^p \, \mathrm dx = \int_{\mathbb R^n} |u(x)|^q \, \mathrm dx \cdot \lambda^{-n}$\\
\item $\int_{\mathbb R^n} |\nabla u_\lambda(x)|^p \, \mathrm dx = \lambda^{p-n} \int_{\mathbb R^n} |\nabla u(x)|^p \, \mathrm dx$.
\end{itemize}
Einsetzen in \eqref{sobolev2} liefert $\| u\|_q \le \lambda^{1-\frac{n}{p} + \frac{n}{q}} \cdot C \| \nabla u \|_p \quad \forall \lambda>0$.

Falls $1-\frac{n}{p} + \frac{n}{q} \neq 0$, dann liefert $\lambda \to 0$ (bzw. $\lambda \to \infty$), dass $\| u\|_q=0$ für alle $u\in C_0^1(\mathbb R^n)$. Ein Widerspruch. Somit ist
\begin{equation}
	1- \frac{n}{p} + \frac{n}{q}=0 \iff \frac{1}{q}=\frac{1}{p} - \frac{1}{n}
\end{equation}
notwendig für die Gültigkeit von \eqref{sobolev1}. 
\begin{df}
Sei $1\le p <n$. Dann ist $p^* >p$ gegeben durch
\begin{equation}
	\frac{1}{p*}= \frac{1}{p} - \frac{1}{n},
\end{equation}
d.h. $p^* = \frac{np}{n-p}$.
\end{df}

\begin{thm}[Gagiardo-Nirenberg-Sobolev]
	Sei $q\le p <n$. Dann exististiert $C=C_{n,p}\in \mathbb R$, so dass gilt
	\begin{equation}
	\|u\|_p^* \le C \| \nabla u\|_p \quad \forall u \in C_0^1(\mathbb R^n).
	\end{equation}
\end{thm}
\begin{rem}
 \begin{itemize}
	\item Die Konstante $C= C_{n,p}$ hängt nicht von $\supp u$ ab, dennoch kann man die Bedingung $\supp u\subset \subset \mathbb R^n$ nicht ersatzlos streichen (vgl $u\equiv 1$).
	\item Unser Beweis liefert $C= \frac{p(n-1)}{n-p}$. Der bestmögliche Wert von $C$ ist jedoch  $C= \sup_u \frac{\|u\|_{p^*}}{\|\nabla u\|_p}$ (dies lässt sich explizit berechnen und nimmt sogar ein Maximum an).
 \end{itemize}
\end{rem}
\begin{proof}
	Sei $p=1$ und somit $p^*= \frac{n}{n-1}$. Sei $U\in C_0^1(\mathbb R^n)$, dann gilt 
	\begin{equation}
		u(x)=u(x_1, \cdots, x_n) = \int_{-\infty}^{x_i} \partial_i u(x_1, \ldots, x_{i-1}, y_i, x_{i+1}, \ldots, x_n) \, \mathrm dy_i \quad (i\in \{1, \ldots, n\}).
	\end{equation}
	und somit 
	\begin{equation}
		|u(x)| \le \int_{-\infty}^\infty | \nabla u(x_1, \ldots, y_i, \ldots, x_n)|\, \mathrm dy_i
	\end{equation}
	bzw.
	\begin{align}
		|u(x)|^{n}{n-1} &\le \left ( \int_{-\infty}^\infty | \nabla u(x)|\, \mathrm dy_i\right ) ^{1/(n-1)}\\
		&\le \prod_{i=1}^n \left ( \int_{-\infty}^\infty |\nabla u(x)|\, \mathrm dx_i \right )^{1/(n-1)}.
	\end{align}
	Wir integrieren beide Seiten bzgl. $x_1$ und verwenden die verallgemeinerte Hölderungleich. Wir bekommen so
	\begin{align}
		\left ( \int_{-\infty}^\infty | u(x)|^{n/(n-1)} \, \mathrm dx_1 \right ) &\le \left ( \int_{-\infty} |\nabla u(x)|\, \mathrm dy_1 \right )^{\frac{1}{n-1}} \cdot \int_{-\infty}^\infty \prod_{i=2}^n \left ( \int_{-\infty}^\infty | \nabla u(x)| \, \mathrm dx \right )
	\end{align}
\end{proof}

(FIXME)

Sei $m>\frac{n}{p}$ bzw $1 > \frac{n}{p}\iff p>n$.
\begin{thm}[Morrey]\label{3.4}
	$n<p\le \infty$ und $u\in C^1(\mathbb R^n)\cap W^{1,p}(\mathbb R^n)$. Dann ist $u$ beschränkt, Hölderstetig mit Exponent $\gamma =1 - \frac{n}{p}$ und
	\begin{equation}
		\| u\|_{\alpha, \gamma} \le C \| u \|_{1,p}
	\end{equation}
	wobei $C$ nur von $n,p$ abhängt.
\end{thm}
\begin{proof}
	$B(x,r) \subset \mathbb R^n$ gilt: (FIXME)
	\begin{equation}
		\not\int_{B(x,r)} | u(y)- u(x) |\, \mathrm dy \le \frac{1}{\omega_n} \int_{B(x,r)} \frac{|\nabla u(x)|}{|y-x|^{n-1}}\, \mathrm dy.
	\end{equation}
	\textbf{Hölderstetigkeit:} Seien $x,y\in \mathbb R^n$ und $r=|x-y|>0$. Sei $W:= B(x,r) \cap B(y,r)$. Für $z\in W$
	\begin{equation}
	|u(y)-u(x)|\le |u(y)-u(z)|+ |u(x)-u(z)|.
	\end{equation}
	Also
	\begin{equation}
		|u(x)-u(y)| \le \not \int_W | u(x) - u(z)| \, \mathrm dz + \not \int_W |u(y)-u(z)|\, \mathrm dz
	\end{equation}
	wobei 
	\begin{align*}
	\not \int_W |u(x)-u(z)|\, \mathrm dz &\le \frac{|B(x,r)|}{|W|} \frac{1}{|B(x,r)|} \int_{B(x,r)} |u(x)- u(z)|\, \mathrm dz\\
	&= C_n \not \int_{B(x,r)} |u(x)-u(z)|\, \mathrm dz \\
	&\le \frac{c_n}{\omega_n} \int_{B(x,r)} \frac{|\nabla u(y)|}{|x-y|^{n-1}} \, \mathrm dy \\
	&\le \frac{c_n}{\omega_n} \left ( \int_{B(x,r)} |\nabla u(y)|^p \, \mathrm dy \right )^{1/p} \left ( \underbrace{\int_{B(x,r)} \frac{1}{|x-y|^{(n-1) p/(p-1)}}\, \mathrm dy}_{=I(r)} \right )^{(p-1)/p}\\
\end{align*}
wobei
\begin{align*}
	I(r) &= \omega_n \int_0^r \, \mathrm dt \, t^{n-1} \frac{1}{t^{(n-1) p/(p-1)}} \\
	&= \omega_n \int_0^r \, \mathrm dt \frac{1}{t^{(n-1)/(p-1)}}\, \mathrm dt = \omega_n r^{1-(n-1)/(p-1)} \frac{1}{1-\frac{n-1}{p-1}} \\
	&= \omega_n r^{(p-n)/(p-1)}\frac{p-1}{p-n}.
\end{align*}
Es folgt
\begin{align*}
	|u(x)-u(y)| &\le c_n \| \nabla u\|_{L^p(B(x,r))} |x-y|^{1-n/p} +\| \nabla u\|_{L^p(B(y,r))} |x-y|^{1-\frac{1}{p}}\\
&\le C_{n,p} |x-y|^{1-\frac{1}{p}}\|\nabla u \|_{L^p(B(x,r)\cup B(x,r))}.
\end{align*}
Somit gilt
\begin{equation}
	\sup_{x\ne y} \frac{|u(x)-u(y)|}{{|x-y|}^{1-\frac{n}{p}}} \le C_{n,p} \| \nabla u\|_{L^p(\mathbb R^n)}.
\end{equation}

\end{proof}

\begin{thm}\label{3.5}
	Sei $\Omega \subset \mathbb R^n$ offen, $n<p<\infty$ und $\gamma = 1- n/p$, dann gilt $W_0^{1,p}(\Omega) \to C^{0,\gamma}(\overline{\Omega})$
\end{thm}
\begin{proof}
	Sei $u\in W_0^{1,p}(\Omega)$ und $(u_n)$ eine Folge in $C_0^\infty(\Omega)$ mit $u_n \to u$ in $W^{1,p}$. Wir setzen $u_n$ durch $0$ zu einer Funktion auf $\mathbb R^n$ fort. Dann gilt nach Theorem \ref{3.4} ...

	Da $C^{0,\gamma}$ ein Banachraum ist, existiert $\tilde u \in C^{0,\gamma}(\mathbb R^n)$ mit $u_n\to \tilde u$ bezechnet $\| \cdot \|_{0,\gamma}$ und insbesondere gleichmäßig. Da $u_n \to u$ in $L^p$ folgt $\tilde u =u$ f.ü. Aus 
	\begin{equation}
		\|u_n\|_{0,\gamma} \le C \|u_n\|_{1,p}
	\end{equation}
folgt im Limes $h\to \infty$
	\begin{equation}
		\|\tilde u \|_{0,\gamma} \le C\| \tilde u\|_{1,p}
	\end{equation}
	wobei $\tilde u$ der stetige Repräsentant von $u$ ist.
\end{proof}
\begin{thm}
	Sei $\Omega \subset \mathbb R^n$ offen, $1\le p < \infty$ und sei $m>\frac{n}{p}$. Dann gilt
	\begin{equation}
	W_0^{m,p}(\Omega) \to C^{k, \gamma}(\overline{\Omega})
	\end{equation}
	für alle $k\in \mathbb N_0$, $\gamma \in (0,1)$ mit 
	\begin{equation}
	m-\frac{n}{p} \ge k + \gamma
\end{equation}
\end{thm}
\begin{rem}
Ist $n/p \not\in \mathbb N$ dann können wir $k=\left [m-\frac{n}{p}\right ] = m- \left [\frac{n}{p}\right ] -1$ und $\gamma = \left ( m- \frac{n}{p} \right )-k = \left [\frac{n}{p} \right ]+1 - \frac{n}{p}$ wählen.

Ist $\frac{n}{p}\in \mathbb N$ dann gilt die Einbettung für $k=m-\frac{n}{p}-1$ und jedes $\gamma \in (0,1)$.
\end{rem}
 \begin{proof}
 \begin{enumerate}
 	\item Falls $p>n$ dann ist $W_0^{m,p}(\Omega) \to C^{m-1, \gamma} (\overline\Omega)$ mit $\gamma = 1- \frac{k}{p}$. (FIXME: überprüfen)
 	\begin{proof}
		Sei $u\in C_0^\infty(\Omega)$. Dann gilt für $|\alpha|\le m-1$ nach Theorem \ref{3.5}.
		\begin{equation}
			\|\partial^\alpha u\|_{0,\gamma} \le C \| \partial^\alpha u\|_{1,p} \le C\| u \|_{m,p}.
		\end{equation}
		Also
		\begin{equation}\label{3.6.1}
			\|u\|_{m-1,p} = \max_{|\alpha|\le m-1} \| \partial^\alpha u \|_\infty + \max_{|\alpha < m-1} | \partial^\alpha u|_\gamma \le C\| u \|_{m,p}
		\end{equation}
		Sei $u\in W_0^{m,p}(\Omega)$, $U_k \in C_0^\infty(\Omega)$ mit $\| u_n - u\|_{m,p} \to 0$. Dann folgt aus \eqref{3.6.1}
		\begin{equation}
			\| u_n - u_m \|_{m-1, \gamma} \le C \| u_n - u_m \|_{m,p} \to 0 \quad (n,m\to \infty)
		\end{equation}
		D.h. $(u_n)$ ist CF in $C^{m-1, \gamma}(\overline{\Omega})$ mit $u_n \to \tilde u$ in $C^{m-1, \gamma}(\overline{\Omega})$. Da $u_n \to u$ in $L^p$ folgt $\tilde u = u$ fast überall und aus
		\begin{equation}
			\| u_n \|_{m-1, \gamma} \le C \| u_n \|_{m,p}
		\end{equation}
		folgt im Limes $n\to \infty$: $\| \tilde u\|_{m-1, r} \le C\| \tilde u \|_{m,p}$.
 	\end{proof}
 \item $W_0^{m,p}(\Omega) \to W_0^{m-l,r}(\Omega)$ falls $\frac{1}{r} = \frac{1}{p} - \frac{l}{n}$ ($l < \frac{n}{p})$.
	\begin{proof}
		Für $|\alpha|\le m-l$, $n\in W_0^{m,p}(\Omega)$ gilt $\partial^\alpha u \in W_0^{l,p}(\Omega) \to L^r(\Omega)$ nach Theorem \ref{3.3}. insbesondere
		\begin{align*}
			\|\partial^\alpha u \| \le C\| \partial ^\alpha u\|_{l,p}&\le C\sum_{|\alpha|\le m-l} \| \partial^\alpha u\|_{l,p}\\
										 &\le C' \sum_{|\alpha | \le m} \| \partial^\alpha u \|_p \le C'' \| u\|_{m,p} 
		\end{align*}
		Somit $\|u\|_{m-l, r} \le \tilde c \| u\|_{m,p}$ für $\tilde c>0$.
	\item Für $p<n$, $\frac{n}{p} \not\int \mathbb N$, gilt die Behauptung des Theorems in der Form der Bemerkung nach dem Theorem
 		\begin{proof}
			Wähle $l\in \mathbb N$ mit
			\begin{equation}
				l < \frac{n}{p} < l+1
			\end{equation}
			d.h. $l= \left [ \frac{n}{p}\right ]$. dann gilt
			\begin{equation}
			\frac{1}{r} = \frac{1}{p} - \frac{l}{n} = \frac{1}{n} \left ( \frac{n}{p} -l\right ) \in (0, \frac{1}{n}).
			\end{equation}
			Also ist $r>n$ und $l < \frac{n}{p} <m$. Aus (2) und (1) folgt also
			\begin{equation}
				W_=^{m,p}(\Omega) \to W_0^{m-l,r}(\Omega) \to C^{m-l-1, \gamma}(\overline{\Omega})
			\end{equation}
			wobei 
			\begin{equation}
				\gamma = 1-\frac{n}{r} = 1-(\frac{n}{p}-l) = l+1 - \frac{n}{p} = [\frac{n}{p}] + 1 - \frac{n}{p}.
			\end{equation}
 		\end{proof}
 	\item Für $p <n$ und $\frac{n}{p} \in \mathbb N$ gilt die Behauptung des Theorems in der Form der Bemerkung:
 		\begin{equation}
 		W^{m,p}(\Omega) \to C^{k,\gamma}(\overline{\Omega}) \quad k=m-\frac{n}{p} -1, \gamma \in (0,1).
 		\end{equation}
 		Beweis beruht auf Theorem \ref{3.7} bzw auf
 		\begin{equation}
 			W_=^{1,n}(\Omega) \to L^q(\Omega) \quad n \le q <\infty.
 		\end{equation}
 		(falls $\Omega$ beschränkt ist braucht man das nicht vgl. Evans.) Wähle $l=\frac{n}{p}-1$. Dann gilt
		\begin{equation}
			\frac{1}{r} = \frac{1}{p} - \frac{l}{n} = \frac{1}{p} - \frac{1}{n} \left ( \frac{n}{p} -1 \right ) = \frac{1}{n}
		\end{equation} (FIXME)
		und somit $r=n$. Nach (2), Theorem \ref{3.7} und (1) gilt
		\begin{equation}
			W_0^{m,p}(\Omega) \to W_0^{m-l,n}(\Omega)\to W_=^{m-l-1,n,q}(\Omega) \to C^{m-\frac{n}{p}-1, \gamma} (\overline \Omega)
		\end{equation}
		wobei $\gamma = 1- \frac{n}{q} \in (0,1)$ und $n<q<\infty$.
\end{proof}
 \end{enumerate}
 \end{proof}
(FIXME) Korrekturen In Theorem 2 und Theorem 3 braucht $\Omega$ nicht beschränkt zu sein! 

\begin{thm}\label{3.7}
 Sei $\Omega\subset \mathbb R^n$ offen und $m=\frac{n}{p}$. Dann gilt
 \begin{equation}
  W_0^{m,p}(\Omega) \to L^q(\Omega)
 \end{equation}
 für alle $q\in [p,\infty)$.
\end{thm}
\begin{proof}
 Das Theorem sei richtig für $m=1$ und $p=n$, d.h.
 \begin{equation}\label{3.7.1}
  W_0^{1,n}(\Omega) \to L^q(\Omega)\quad q \in [n,\infty)
 \end{equation}
Dann gilt für $m=\frac{n}{p}>1$ nach Theorem \ref{3.3}
\begin{equation}
 W_0^{m,p}(\Omega) \to W_0^{1,r}
\end{equation}
\begin{equation}
 \frac{1}{r} = \frac{1}{p} - \frac{m-1}{n} = \frac{1}{n} \left ( \frac{n}{p} -m+1 \right )= \frac{1}{n}
\end{equation}
für $q\in [n,\infty)$. Da $W_0^{m,p}(\Omega)\to L^q(\Omega)$ (FIXME)

$n=1$: $W_0^{1,1}(\Omega) \to L^\infty(\Omega)$ (Blatt 3) also gilt sogar $W_0^{1,1}(\Omega) \to L^q(\Omega)$ 
für alle $u\in C_0^\infty(\Omega)$.
\begin{equation}\label{3.7.2}
 \left ( \int |u|^{\gamma n/(n-1)} \, \mathrm dx \right )^{(n-1)/n} \le \gamma \left ( \int |u|^{(\gamma-1)n/(n-1)}\, \mathrm dx\right )^{(n-1)/n} (\int |\nabla u|^n \, \mathrm dx)^{1/n} \quad \gamma>1
\end{equation}
(FIXME)
Wir zeigen induktiv, dass
\begin{equation}
 W_0^{1,p}(\Omega) \to L^{q_j}(\Omega), \quad q_j = \frac{(n+j-1)n}{n-1}\quad j=0,1,2,\ldots.
\end{equation}
\textbf{Verankerung:} 
\begin{equation}
 W_0^{1,n}(\Omega) \to L^n(\Omega) = L^{q_0}(\Omega)
\end{equation}
\textbf{Schritt:} Angenommen $W_0^{1,n}(\Omega) \to L^{q_j}(\Omega), j\ge 0$. Wähle $\gamma$ so, dass 
\begin{equation}
 (\gamma-1) \frac{n}{n-1} = q_j = \frac{(n+i-1)n}{n-1}. 
\end{equation}
D.h. $\gamma=n+j$. Dann $\gamma \frac{n}{n-1} = (n+i) \frac{n}{n-1} = q_{i+1}$ und somit nach  \eqref{3.7.2}
\begin{equation}
 (\int |u|^{q_j+1} \, \mathrm dx)^{(n-1)/n} \le (n+j) ( \int |u|^{q_j}\, \mathrm dx)^{(n-1)/n} \| \nabla u\|_n
\end{equation}
d.h.
\begin{equation}
 \|u\|_{q_{j+1}}^{q_{i+1} (n-1)/n} \le ( n+i) \| u\|_{q_i}^{q_i \frac{n-1}{n}} \| \nabla u\|_n 
\end{equation}
oder
\begin{equation}
 \| u\|^{n+i}_{q_{i+1}} \le (n+i) \| u\|_{q_i}^{q_i (n-1)/n}\| \nabla u\|_n
\end{equation}
Dann folgt
\begin{align*}
 \|u\|_{q_{i+1}} &\le (n+j)^{q_i (n-1)/n}\| \nabla u\|_n\\
 &\le (n+j)^{1/(n-i)}\\ % vor allem hier
 &\le c_{n,j} \| u \|_{1,n}
\end{align*}(FIXME)
nach Induktionsannahme. Durch das übliche Approximationsargument ($C_0^\infty(\Omega) \subset L^{q_j}(\Omega)$ dicht) folgt nun $W^{1,n}(\Omega) \to L^{q_{i+1}}(\Omega)$.

Da $q_j \to \infty$ ($j\to \infty$) und $W_0^{1,n}(\Omega) \to L^n(\Omega)$ folgt die Behauptung \eqref{3.7.1}.
\end{proof}
\begin{thm}[Rellich-Kondrachov]
 Sei $\Omega \subset \mathbb R^n$ offen und beschränkt. Sei $p\le n$. Dann ist die Einbettung 
 \begin{equation}
  W_0^{1,p}(\Omega) \to L^q(\Omega) \quad (1\le q < p^*)
 \end{equation}
 kompakt wobei
 \begin{align*}
  \frac{1}{p^*} = \frac{1}{p} - \frac{1}{n} &\ \quad p<n\\
  p^*=\infty &\ \quad p=n
 \end{align*}
\end{thm}
\begin{proof}
Sei zuerst $p<n$. Wir zeigen zuerst, dass $W_0^{1,1}(\Omega) \to L^1(\Omega)$ kompakt. Da $\Omega$ beschränkt ist, gilt $W_0^{1,p}(\Omega) \to W_0^{1,1}(\Omega)$ und somit $W_0^{1,p}(\Omega) \to L^1(\Omega)$ kompakt. Sei $A=\{u\in W^{1,1}(\Omega) |\|u\|_{1,1} \le 1\}$. Wir zeigen, dass $\overline{A}^{\|\cdot\|_1}$ in $L^1(\Omega)$ kompakt ist. Sei $A_\varepsilon = \{u_\varepsilon\big |_\Omega |u\in A\}$ wobei
\begin{equation}
 u_\varepsilon = (J_\varepsilon * u) \quad J_\varepsilon(x) = \varepsilon^{-n} J( \frac{x}{\varepsilon}), \|J\|_1=1. 
\end{equation}
\textbf{Schritt 1:} Für jedes $\varepsilon >0$ ist $\overline{A_{\varepsilon}}^{\|\cdot\|_1}$ kompakt in $L^1(\Omega)$.
\begin{proof}
Es gilt für $u\in A$:
\begin{align*}
 |u_\varepsilon(x)| &= |\int J_\varepsilon(x-y) u(y) \, \mathrm dy| \\
 &\le \| J_\varepsilon \|_\infty \| u \|_1 \le \varepsilon^{-n} \| J\|_\infty\|u\|_1
\end{align*}
und 
\begin{align*}
 |\nabla u_\varepsilon(x)| &\le \int |\nabla J_\varepsilon(x-y)||u(y)|\, \mathrm dy\\
 &\le \| \nabla J_{\varepsilon}\|_\infty \cdot \|u\|_1 \le c \varepsilon^{-n-1}.
\end{align*}
Somit ist $A_\varepsilon$ gleichmäßig beschränkt und gleichgradig stetig und somit ist
$\overline{A_\varepsilon}^{\|\cdot \|_\infty}$ kompakt in $C(\overline{\Omega})$ (Arzela-Acoli). Daraus folgt dass $\overline{A_\varepsilon}^{\| \cdot \|_1}$ kompakt ist in $L^1(\Omega)$.
(\textbf{Beweis:} Sei $(u_n)$ eine Folge in $\overline{A_\varepsilon}^{\|\cdot \|_1}$. Dann existiert $(v_n)$ in $A_\varepsilon$ mit $\| u_n - v_n\|_1 < \frac{1}{n}$ Da $\overline{A_\varepsilon}^{\|\cdot \|_\infty}$ kompakt ist, existieren Teilfolgen $(v_{n_k})$ und $v\in C(\overline{\Omega})$ mit $\| v_{n_k} -v\|_\infty \to 0$. $\overline \Omega$ beschränkt, so folgt $v\in L^1(\overline{\Omega})$ underbrace
\begin{align*}
 \|u_{n_k}-v\|_{L^1(\Omega)} &\le \| v_{n_k}-v\|_{L^1(\Omega)} + \frac{1}{n_k}\\
 &\le c\| v_{n_k}-v\|_\infty + \frac{1}{n_k} \to 0 \quad (k\to \infty)
\end{align*}
\end{proof}
\textbf{Schritt 2:} Für alle $u\in A$ gilt
\begin{equation}
 \|u_\varepsilon -u\|_{L^1(\Omega)} \le \varepsilon.
\end{equation}
\begin{proof}
Sei zuerst $u\in C_0^\infty(\Omega) \cap A$. Dann gilt
\begin{align*}
 u_\varepsilon(x) &= \int J_\varepsilon(x-y) u(y) \, \mathrm dy \\
 &= \int \varepsilon^{-n} J(\frac{y}{\varepsilon}) u(x-y)\, \mathrm dy \quad z=\frac{y}{\varepsilon}\\
 &= \int J(z) u(x-\varepsilon z) \, \mathrm dz.
\end{align*}
\begin{align*}
 |u_\varepsilon(x) -u(x)| &= | \int J(z) ( u(x-\varepsilon z) - u(x)) \, \mathrm dz| \\
 &\le \int J(z) |u(x- \varepsilon z) - u(x)|\, \mathrm dz
\end{align*}
wobei
\begin{align*}
 |u(x-\varepsilon z) - u(x) | &\le \int_0^1 |\nabla u(x-t\varepsilon z) | |\varepsilon z|\, \mathrm dt\\
 &= \varepsilon |z| \int_0^1 |\nabla u(x-t\varepsilon z) |\, \mathrm dt.
\end{align*}
Also
\begin{equation}
 \int | u_\varepsilon(x) - u(x)|\, \mathrm dx\le \varepsilon \int_{|z|\le 1} \mathrm dz J(z) |z| \int |\nabla u(x)|\, \mathrm dx \le \varepsilon \| \nabla u\|_1\le \varepsilon \| u\|_1\le \varepsilon.
\end{equation}
(FIXME)
Sei nun $u\in A$. Dann existiert $\phi \in C_0^\infty(\Omega)$ mit $\|u-\phi\|_1 <\delta$ und $\| \phi\|_{1,1} \le 1$ denn $C_0^\infty(\Omega) \subset L^1(\Omega)$ dicht. Dann gilt
\begin{equation}
 \|J_\varepsilon* \phi - \phi \|_1 \le \varepsilon.
\end{equation}
Also 
\begin{align*}
 \| J_\varepsilon*u -u\|_1 &\le \underbrace{\|J_\varepsilon * \phi - \phi\|_1}_{\le \|u-\phi\|_1} + \| J_\varepsilon Ü (u-\phi)\|_1 + \| u-\phi\|_1\\
 &\le \varepsilon + 2\delta.
\end{align*}
Da $\delta >0$ belibig war, folgt
\begin{equation}
 \| J_\varepsilon * u - u \|_1 \le \varepsilon.
\end{equation}
Wir zeigen jetzt, dass $A$ total beschränkt ist in $L^1(\Omega)$. Dann ist $\overline{A}^{\|\cdot\|_1}$ total beschränkt und abgeschlossen also kompakt in $L^1(\Omega)$.

Wir zeigen: $\forall \varepsilon >0 \exists N \in \mathbb N$ und $u_1, \ldots, u_n \in A$ mit
\begin{equation}
 A \subset \bigcup_{i=1}^N B(u_i, \varepsilon)
\end{equation}
(Kugeln in $L^1$-Norm). Wir wissen, dass $\overline{A_\varepsilon}^{\|\cdot \|_1}$ kompakt ist. Aus
\begin{equation}
 \overline{A_\varepsilon}^{\|\cdot\|_1} \subset\bigcup_{U\in A} B(u_\varepsilon, \varepsilon)
\end{equation}
folgt dass
\begin{equation}
 \overline{A_\varepsilon}^{\|\cdot \|_1} \subset \bigcup_{i=1}^N B(u_{i,\varepsilon}, \varepsilon)
\end{equation}
für $u_1,\ldots, u_N \in A$. Nach Schritt 2 ist $\| u_{i,\varepsilon} - u_i \| \le \varepsilon$. Also
\begin{equation}
 B(u_{i,\varepsilon}, \varepsilon) \subset B(u_i, 3\varepsilon)
\end{equation}
und aus $u\in A$ folgt $u_\varepsilon \in A_\varepsilon$ wobei $\| u_\varepsilon - u\| \le \varepsilon$. Also
\begin{equation}
 A\subset \bigcup_{i=1}^N B(u_i, \psi_\varepsilon).
\end{equation}
Also ist $A$ total beschränkt. Wir haben also gezeigt, dass $W_0^{1,1}(\Omega) \to L^1(\Omega)$ kompakt für $p<n$.

Sei $p>1$.  Da $\Omega$ beschränkt ist, ist $L^p(\Omega)\to L^1(\Omega)$ stetig und somit auch 
$W_0^{1,p}(\Omega) \to W_0^{1,1}(\Omega)$ stetig. Also ist $W_0^{1,p}(\Omega) \to L^{1}(\Omega)$ kompakt. Im Fall $p<n$ ist $W_0^{1,p}(\Omega) \to L^{p^*}(\Omega)$ stetig und für $1\le q<p^*$ gilt
\begin{equation}
 \frac{1}{q} = \theta + \frac{1-\theta}{p^*} \quad \text{ mit } \theta\in (0,1]
\end{equation}
und 
\begin{equation}
 \| u\|_q \le \|u\|_1 \| u\|_{p^*}
\end{equation}
für $u\in W_0^{1,p}(\Omega)$.  Sei $(u_k)$ eine beschränkte Folge in $W_0^{1,p}(\Omega)$, z.B. $\|u_k\|_{1,p}\le M$. Dann folgt
\begin{equation}\label{3.8kompakt}
 \|u_k - u_l\| \le c \|u_k - u_l\|_1 (2M)^{1-\varepsilon}.
\end{equation}
Da $W_0^{1,p}(\Omega) \to L^1(\Omega)$ kompakt ist, enthält $u_k$ eine $L^1$-Cauchyfolge. Diese sei auch mit $(u_k)$ bezeichnet. Dann zeigt \eqref{3.8kompakt}, dass $(u_k)$ eine $L^q$-Cauchyfolge ist. Also ist $W_0^{1,p}(\Omega) \to L^q(\Omega)$ kompakt.

Im Fall $p=n>1$ ist $1\le p - \varepsilon <n$ für geeignetes $\varepsilon>0$ und daher ist 
\begin{equation}
 W_0^{1,p}(\Omega) \stackrel{\text{stetig}}{\to} W_0^{1,p-\varepsilon}(\Omega) \stackrel{\text{kompakt}}{\to} L^q(\Omega)
\end{equation}
kompakt für $1\le q < \frac{n(p-\varepsilon)}{n-(p-\varepsilon)} = \frac{n(n-\varepsilon)}{\varepsilon}$.
\end{proof}
\begin{rem}
 Für $p>n$ ist $W_0^{1,p}(\Omega) \to L^q(\Omega)$ kompakt für $1\le q \le\infty$ denn 
 \begin{equation}
  W_0^{1,p}(\Omega) \to C^{0,\gamma}(\overline{\Omega}), \quad \gamma = 1-\frac{n}{p}.
 \end{equation}
 stetig sowie 
 \begin{equation}
  C^{0,\gamma}(\overline{\Omega}) \to C^0(\overline{\Omega})
 \end{equation}
kompakt und
 \begin{equation}
  C^0(\overline{\Omega}) \to L^q(\Omega)
 \end{equation}
 stetig.

 Wir wollen nun Einbettungssätze für $W^{m,p}(\Omega)$ beweisen. Diese führen wir auf Einbettungssätze für $W^{1,p}(\Omega)$ zurück.  

 Die Idee ist die Funktion $u\in W^{1,p}(\Omega)=W_0^{1,p}(\mathbb R^n)$ fortzusetzen zu Funktionen $\tilde u \in W^{1,p}(\tilde \Omega)$ mit $\tilde u = u$ in $\Omega\subset \tilde \Omega$.   Dann können wir die bekannten Sätze auf $W_0^{1,p}(\tilde \Omega)$ anwenden.   Genauer suchen wir eine beschränkte lineare Abbildung $E: W^{1,p}(\Omega)\to W_0^{1,p}(\tilde \Omega)$ mit $Eu=u$ in $\Omega$.  Die Existenz von $E$ impliziert dass $C^\infty(\overline{\Omega}) \cap W^{1,p}(\Omega)$ in $W^{1,p}(\Omega)$ dicht ist.
 
 \textbf{Erinnerung:} Nach Meyers-Serrin ist $C^\infty(\Omega) \cap W^{1,p}(\Omega)$ dicht in $W^{1,p}(\Omega)$ wobei $C^\infty(\Omega) \supset C^\infty(\overline{\Omega})$.

Diese Dichtheit brauchen wir für die Konstruktion von E:
\end{rem}

\begin{df}
 Der Rand von $\Omega\subset \mathbb R^n$ ist von der Klasse $C^\infty$ falls zu jedem $x_0 \in \partial \Omega$ eine Umgebung $U\subset \mathbb R^n$ und eine stetige Abbildung
 \begin{equation}
  h: B(0, \eta) \to \mathbb R \quad B(0,\eta) \subset \mathbb R^{n-1}
 \end{equation}
existiert so, dass (nach Verschiebung und Rotation des Koordinatensystems).
\begin{equation}
 \Omega \cap U = \{ (x', x_n) \in U| x_n >h(x') , x'\in B(0, \eta)\}
\end{equation}
\end{df}
\begin{thm}\label{thm3.9}
 Ist $\Omega \subset \mathbb R^n$ offen, beschränkt und $\partial \Omega$ von der Klasse $C^0$, dann ist $C^\infty(\overline{\Omega}) \cap W^{m,p}(\Omega)$ dicht in $W^{m,p}(\Omega)$ für $m\in \mathbb N$, $1\le p <\infty$.
\end{thm}

\begin{proof}
Da $\partial \Omega$ kompakt ist, existieren $U_1,..., U_N \subset \mathbb R^n, N<\infty$ mit
\begin{equation}
 \partial \Omega \subset \bigcup_{i=1}^N U_i
\end{equation}
so dass $\partial\Omega \cap U_i$ (nach Verschiebung un Rotation) als Graph einer stetigen Funktion $h_i$ dargestellt werden kann. Sei $U_0\subset \Omega$ offen mit $U_0\supset \Omega \setminus (\bigcup_{i=1}^N U_i)$ so dass
\begin{equation}\label{3.9kompakt}
 \overline{\Omega} \subset \bigcup_{i=0}^N U_i.
\end{equation}
Sei $\phi_0, \phi_1,\ldots, \phi_N \in C_0^\infty(\mathbb R^n)$ eine der offenen Überdeckung $(*)$ untergeordnete Zerlegung der Eins, d.h.
\begin{equation}
 \supp(\phi_i) \subset U_i \quad \sum_{i=0}^N \phi_i =1 \quad{auf } \overline{\Omega}
\end{equation}
(vgl. Satz 2.5).

Sei $u\in C^\infty(\Omega) \cap W^{m,p}(\Omega)$. Wir wollen $u$ approximieren durch Elemente aus $C^\infty(\overline{\Omega})\cap W^{m,p}(\Omega)$.  Sei $u_i = \phi_i u, i=0,\ldots, N$.  Dann ist $\phi_0 \in C_0^\infty(\Omega) \subset C^\infty(\overline{\Omega})$ und 
\begin{equation}
 \sum_{i=0}^N u_i = u, \supp(u_i) \subset U_i
\end{equation}
Für $i=1,\ldots, N$ setzen wir $u_i$ durch $0$ auf $\mathbb R^n$ fort und definieren
\begin{equation}
 u_{i,\tau}(x) =u_i(x+ \tau e_n) \quad \tau >0
\end{equation}
wobei $e_n=(0,\ldots,0,1)$.

FIXME(BILD?)

$u_i$ ist $C^\infty$ in 
\begin{equation}
U_i \cap \Omega = \{(x', x_n) \in U_i | x_n > h(x'), x' \in B(0,\eta_i)\}.
\end{equation} 
$u_i$ ist $C^\infty$ in 
\begin{equation}
 \{(x', x_n) \in U_i | x_n > h(x') - \tau \quad x'\in B(0,\eta_i)\}
\end{equation}
was eine offene Umgebung der kompakten Menge $\supp(u_{i,\tau}) \cap \overline{\Omega}$. Also ist $u_{i,\tau}\in C^\infty(\overline{\Omega})$. Es folgt $u_0 + \sum_{i=1}^N u_{i,\tau} \in C^\infty(\overline{\Omega})$ wobei
\begin{align*}
 \| u_0 + \sum_{i=1}^N u_{i,\tau} -u\|_{m,p} &= \| \sum_{i=1}^N (u_{i,\tau} - u_i)\|_{m,p}\\
 &\le \sum_{i=1}^N \| u_{i,\tau} - u_i \|_{m,p, \Omega} \to 0 (\tau \to 0).
\end{align*}
Wenn
\begin{equation}
 \| u_{i,\tau} - u_i \|^p_{m,p, \Omega} = \sum_{|\alpha|\le m} \int_\Omega | \partial^\alpha u_i (x+ \tau e_n) - \partial^\alpha u_i (x)|^p \to 0 (\tau \to 0). 
\end{equation}
Da $C^\infty(\Omega) \cap W^{m,p}(\Omega)$ dicht ist in $W^{m,p}(\Omega)$ folgt daraus die Behauptung des Theorems.







\end{proof}


\end{proof}




\begin{thebibliography}{xxx}
\bibitem[AF]{AF} Robert A. Adams and John F.  Fournier, \textit{Sobolev Spaces}, 2nd Edition, Academic Press (2003).
\end{thebibliography}
\end{document}



